\documentclass[12pt,a4wide]{report}
\usepackage[utf8]{inputenc}
\usepackage{amsmath}
\usepackage{amsfonts}
\usepackage{amssymb}
\usepackage{color}
\usepackage{hyperref}
\usepackage{listings}
\usepackage{palatino}
\usepackage{parskip}

\newcommand{\superscript}[1]{\ensuremath{^{\textrm{#1}}}}
\newcommand{\subscript}[1]{\ensuremath{_{\textrm{#1}}}}

%\newcommand{\th}[0]{\superscript{th}}
\newcommand{\st}[0]{\superscript{st}}
\newcommand{\nd}[0]{\superscript{nd}}
\newcommand{\rd}[0]{\superscript{rd}}

\newcommand{\amp}{{\fontfamily{ppl}\selectfont\emph\&}}

\lstset{numbers=left, basicstyle=\ttfamily, breaklines=true}

\begin{document}
	\title{Sentiment Analysis, News \amp{} CAC40}
	\date{\today}
	\author{Alexis Daboville}
	
	\maketitle
	
	\pagenumbering{roman}
	\pagestyle{headings}
	\tableofcontents
	
	\chapter{Acknowledgements}
	
	\chapter{Introduction}
		% what? sentiment analysis? CAC?
		% why ?
		% how ?
		% hypothesis
	
	\chapter{Literature Review}
	
	\chapter{Methodology}
		\section{Getting the news}
		
			There was a few options to consider for getting news data. The first one was to know where to get the data, the second was how to get it.
			
			In 2009, Stéphane Kazmierczak did similar experiments and got his data by crawling the website Investir.fr\footnote{\url{http://www.investir.fr/}}. Investir.fr is a website about economy and stock markets. The problem is that there are two downsides using this approach:
			\begin{itemize}
				\item All the data comes from a single source.
				\item Currently most of economics news website have paid archives (which prevented me from crawling them).
			\end{itemize}
			
			The solution proposed by Khurshid Ahmad was to get the data from Lexis Nexis\footnote{\url{http://www.lexisnexis.com/en-us/home.page}}. Lexis Nexis is a platform which has numerous archives of newspapers (including French economic newspapers). Fortunately Trinity College has an agreement with them which allow Trinity students to use the service. Using this platform solved the two problems. In fact I was able to get data from multiple newspapers, and the Lexis Nexis website is ``crawlable''.
			
			However searching for archives for more than a few days is a tedious task. In consequence I developed a web crawler which browse Lexis Nexis automagically and download the data. Though, there was a few considerations to take into account. The first one was to crawl Lexis Nexis slowly to avoid to be ban from using the service. The second one was how to crawl the website. The first problem was easily solved, I decided to crawl no more than one month of data per day. So I needed roughly two month to get 64 months of data. The technical problem was more interesting. The first thing to consider is that I needed to be able to perform a lot of things with my robot: login, search for archives of each day by filling multiple forms, clicking on various links\ldots. At first sight I though I would use Perl WWW::Mechanize\footnote{\url{http://search.cpan.org/dist/WWW-Mechanize/}} or Python Mechanize\footnote{\url{http://wwwsearch.sourceforge.net/mechanize/}}. Unfortunately none of them was able to handle the Javascript stuff (and Lexis Nexis heavily rely on the ability to execute Javascript). In consequence I quickly look on the internet to find an alternative, and I found Watir\footnote{\url{http://watir.com/}}. Watir is described by Wikipedia\footnote{\url{http://en.wikipedia.org/wiki/Watir}} like that ``\emph{Web Application Testing in Ruby (or Watir, [\ldots]) is a toolkit used to automate browser-based tests during web application development.}''. So even if it's primary designed for testing purposes we can use it to crawl a website. In fact Watir was the chosen solution because it was simple, and there wasn't any better alternative. Moreover Watir is really simple to use, the basic steps for using it are:
			\begin{enumerate}
				\item Find the correct HTML element.
				\item Set it to the right value and/or click it.
			\end{enumerate}
			
			For instance here's the method (part of the \lstinline!LexisNexisCrawler! class, in the file \lstinline!tools/lncrawler.rb!) that log the user in:
			
			\begin{lstlisting}[language=Ruby, basicstyle=]
def login(url, user='', pass='')
  # ask for login details
  # before going to login
  # page, so once it's filled user can "go away"
  # obviously if login details are 
  # provided to the method, doesn't ask
  while user.empty? or pass.empty?
    puts 'Username:'
    user = gets.chomp
    puts 'Password:'
    pass = gets.chomp
  end

  # goto login page
  @browser.goto url

  @browser.text_field(:name => 'user').set(user)
  @browser.text_field(:name => 'pass').set(pass)
  @browser.button(:name => 'Submit2').click
                                                                                                 
  # accept term & conditions
  @browser.link(:href => /submitterms\.do/).click

  # get the search link
  @searchUrl = @browser.link(:text => 'Search').href
end
			\end{lstlisting}
			
			Using the crawler developed with Watir I was able to fetch 560,000+ news. The time frame spans from January 1\st, 2004 to March 31\st, 2009. Also the data was collected from several French economic newspapers or websites:
			\begin{itemize}
				% + footnote with url to each website?
				\item L'Agefi quotidien
				\item Boursier.com
				\item Les \'Echos
				\item Les\'Echos.fr
				\item Le Figaro économie
				\item Investir.fr
				\item Le Parisien économie
				\item La Tribune
				\item LaTribune.fr
			\end{itemize}
			
		\section{Getting the CAC40 data}
		
			The CAC40 data was simply downloaded from Yahoo! Finance\footnote{\url{http://finance.yahoo.com/q/hp?s=^FCHI+Historical+Prices}} which provides all the data I needed in a handy  CSV file.
		
		\section{The dictionaries}
		
			To analyse the sentiments expressed in the news I also needed two dictionaries. The first one would contain the positive or negative words, and the second would contain the words related to economy.
			
			For the first dictionary, the idea used was to merge two things:
			\begin{itemize}
				\item The General Inquirer (GI) Dictionary\footnote{\url{http://www.wjh.harvard.edu/~inquirer/}} which is the reference dictionary for sentiment analysis. It classifies hundreds of words in several categories like strong or weak, positive or negative, active or passive\ldots Here we just need the positive and negative categories.
				\item The most common words in economic news.
			\end{itemize}
			
			The GI dictionary was translated in French by Stéphane Kazmierczak, so I used it as a base.
			
			The most commons words were simply found using a little Python script. It just splitted all text in different words (in Python terms it means \lstinline!re.findall(r'\w+', text, re.UNICODE)!), and finally just use \lstinline!collections.Counter! and the \lstinline!.most_common()! method to find the most commons words.
			
			I merged the two dictionaries by hand because I wanted to add a few improvements over the Kazmierczak's one. The first thing was to only keep words that were often used in French. The second and more important improvement was to allow jokers at the end of the words. For instance instead of storing the words \emph{augmentation}, \emph{augmenter}, \emph{augmente} and so on I used a * as a joker and we just store \emph{augment*}, so all words which have the prefix \emph{augment} are matched.
			
			The second dictionary was easy to build, I just translated (some of)\footnote{Only the ones that were translatable by a single word in French.} the economic words listed on InvestorWords.com\footnote{\url{http://www.investorwords.com/cgi-bin/bysubject.cgi?8}}.
			
			Then when I need to lookup if a word is in one of the dictionary I use tries\footnote{Also known as prefix tree, see \url{http://en.wikipedia.org/wiki/Trie} for more informations.}, so the lookup is fast.
			
		\section{Processing the data}
		
			Once I had all the news the first thing to do was to reformat them. For example I had to remove useless metadata and normalize the dates (which were sometimes in French and sometimes in English\ldots). This work was done with a little Python script (\lstinline!tools/format.py!).
			
			Once it was done I was able to perform calculation over the corpus. For each day I computed the number of positive words, negative words and the total number of words.
			
			\begin{eqnarray*}
			words(day) &=& \textrm{A set of all words in the news of day}\\
			\\			
			positive\_set &=& positive\_words\_set - economic\_words\_set\\
			positive\_words(day) &=& \sum_{w \textrm{ in } words(day)} (1\textrm{ if }w \textrm{ in } positive\_set \textrm{ otherwise } 0)\\
			\\
			negative\_set &=& negative\ words\ set - economic\ words\ set\\
			negative\_words(day) &=& \sum_{w \textrm{ in } words(day)} (1\textrm{ if }w \textrm{ in } negative\_set \textrm{ otherwise } 0)\\
			\\
			number\_of\_words(day) &=& length(words(day))
			\end{eqnarray*}
			
			For better results, the function words splitted news using a regex. Then I used a function to remove all diacritical signs (so the words were kind of ``normalized'', for instance the word \emph{éléphant} is transformed to \emph{elephant}. This is done by the following Python code:
			
			\begin{lstlisting}[language=Python]
def normalize(w):
    """Convert an UTF-8 encoded string to a pure ASCII one.

    Also call lower on the word."""

    nfkd = unicodedata.normalize('NFKD', w)
    return ''.join(x for x in nfkd if unicodedata.category(x)[0] == 'L').lower()


_words_re = re.compile(r'\w+')
def words(s):
    """Returns all the words of a string."""

    return (normalize(w) for w in re.findall(_words_re, s))
    			\end{lstlisting}
    			
    		The final step of processing the data was to merge the CSV produced by the script which count the words (\lstinline!tools/count.py!) and the CAC40 data. This was done by another Python script (\lstinline!tools/merge.py!).
			
		\section{Analysis of the data}
			% libreoffice calc	
			
	\chapter{Experiments and Evaluation}
		\section{Graph analysis}
		
		\section{Correlation between news and the CAC}
	
	\chapter{Afterword}
		% conclusion
		% results?
		% future work
		
\end{document}
